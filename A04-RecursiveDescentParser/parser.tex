\documentclass[12pt,letterpaper]{article}
%\usepackage[utf8]{inputenc}
\UseRawInputEncoding
\usepackage[english]{babel}
\usepackage{listings}
\usepackage{xcolor}
\usepackage{graphicx}

%For syntax highlighting
\definecolor{codegreen}{rgb}{0,0.6,0}
\definecolor{codegray}{rgb}{0.5,0.5,0.5}
\definecolor{codepurple}{rgb}{0.58,0,0.82}
\definecolor{backcolour}{rgb}{1,1,1}

%%Sets different parameters
\lstdefinestyle{mystyle}{
	backgroundcolor=\color{backcolour},   
    commentstyle=\color{codegreen},
    keywordstyle=\color{magenta},
    numberstyle=\tiny\color{codegray},
    stringstyle=\color{codepurple},
    basicstyle=\ttfamily\footnotesize,
    breakatwhitespace=false,         
    breaklines=true,                 
    captionpos=b,                    
    keepspaces=true,                 
    numbers=left,                    
    numbersep=5pt,                  
    showspaces=false,                
    showstringspaces=false,
    showtabs=false,                  
    tabsize=4
}
\lstset{style=mystyle}

\title{\textbf{Department of Computer Science and Engineering}}
\author{\textbf{S.G.Shivanirudh , 185001146, Semester VI }}

\date{26 February 2021}

\begin{document}
\maketitle
\hrule
\section*{\center{UCS1602 - Compiler Design}}
\hrule 
\bigskip\bigskip

%Assignment name
\subsection*{\center{\textbf{Exercise 4: Recursive Descent Parser using C}}}

%Objective
\subsection*{\flushleft{Objective:}}
\begin{flushleft}
    Write a program in C to construct Recursive Descent Parser for the following grammar which is for arithmetic expression involving + and   *.
    Check the   Grammar   for   left   recursion   and   convert   into suitable for this parser. 
    Write recursive functions for every non-terminal.   Call   the   function   for   start   symbol   of   the   Grammar   in main().
    Extend this parser to include division, subtraction and parenthesis operators
\end{flushleft}

%Code
\subsection*{\flushleft{Code:}}
\begin{flushleft}
\lstinputlisting[language = C, firstline = 1, lastline = 208]{parserext.c}
\end{flushleft}

\newpage
%Output
\subsection*{\flushleft{Output:}}
\subsubsection*{\flushleft{Success scenario:}}
\begin{figure}[h]
    \centering
    \includegraphics[width = \textwidth]{Pics/Success1.png}
\end{figure}
\begin{figure}
    \centering
    \includegraphics[width = \textwidth]{Pics/Success2.png}
\end{figure}
\begin{figure}
    \centering
    \includegraphics[width = \textwidth]{Pics/Success3.png}
\end{figure}

\newpage
\subsubsection*{\flushleft{Failure scenario:}}
\begin{figure}[h!]
    \centering
    \includegraphics[width = \textwidth]{Pics/Failure1.png}
\end{figure}
\newpage
\begin{figure}[h]
    \centering
    \includegraphics[width = \textwidth]{Pics/Failure2.png}
\end{figure}
\hrule
\end{document}