\documentclass[12pt,letterpaper]{article}
%\usepackage[utf8]{inputenc}
\UseRawInputEncoding
\usepackage[english]{babel}
\usepackage{listings}
\usepackage{xcolor}
\usepackage{graphicx}

%For syntax highlighting
\definecolor{codegreen}{rgb}{0,0.6,0}
\definecolor{codegray}{rgb}{0.5,0.5,0.5}
\definecolor{codepurple}{rgb}{0.58,0,0.82}
\definecolor{backcolour}{rgb}{1,1,1}

%%Sets different parameters
\lstdefinestyle{mystyle}{
	backgroundcolor=\color{backcolour},   
    commentstyle=\color{codegreen},
    keywordstyle=\color{magenta},
    numberstyle=\tiny\color{codegray},
    stringstyle=\color{codepurple},
    basicstyle=\ttfamily\footnotesize,
    breakatwhitespace=false,         
    breaklines=true,                 
    captionpos=b,                    
    keepspaces=true,                 
    numbers=left,                    
    numbersep=5pt,                  
    showspaces=false,                
    showstringspaces=false,
    showtabs=false,                  
    tabsize=4
}
\lstset{style=mystyle}

\title{\textbf{Department of Computer Science and Engineering}}
\author{\textbf{S.G.Shivanirudh , 185001146, Semester VI }}

\date{5 March 2021}

\begin{document}
\maketitle
\hrule
\section*{\center{UCS1602 - Compiler Design}}
\hrule 
\bigskip\bigskip

%Assignment name
\subsection*{\center{\textbf{Exercise 5: Implementation of Desk Calculator using YaccTool}}}

%Objective
\subsection*{\flushleft{Objective:}}
\begin{flushleft}
    Write Lex program to recognize relevant tokens required for the Yacc parser to implement desk calculator. 
    Write the Grammar for the expression involving the operators.
    Precedence and associativity has to be preserved. Yacc is available as a command in linux. 
    The grammar should have non-terminals E, Op and a terminal id.
\end{flushleft}

%Code
\subsection*{\flushleft{Code:}}
\subsubsection*{\flushleft{Lex:}}
\begin{flushleft}
\lstinputlisting{calc.l}
\end{flushleft}
\subsubsection*{\flushleft{Yacc:}}
\begin{flushleft}
\lstinputlisting{calc.y}
\end{flushleft}

\newpage
%Output
\subsection*{\flushleft{Output:}}
\subsubsection*{\flushleft{Arithmetic Expression:}}
\begin{figure}[h]
    \centering
    \includegraphics[width = \textwidth]{Pics/AE.png}
\end{figure}

\newpage
\subsubsection*{\flushleft{Boolean Expression:}}
\begin{figure}[h]
    \centering
    \includegraphics[width = \textwidth]{Pics/BoolExp1.png}
\end{figure}
\begin{figure}[h]
    \centering
    \includegraphics[width = \textwidth]{Pics/BoolExp2.png}
\end{figure}

\newpage
\subsubsection*{\flushleft{Bitwise operation:}}
\begin{figure}[h]
    \centering
    \includegraphics[width = \textwidth]{Pics/BitExp1.png}
\end{figure}
\begin{figure}[h]
    \centering
    \includegraphics[width = \textwidth]{Pics/BitExp2.png}
\end{figure}
\hrule
\subsection*{\flushleft{Learning Outcomes:}}
\begin{flushleft}
    \renewcommand{\labelitemi}{$\textendash$}
    \begin{itemize}
        \item Understood the basic working of Yacc tool. 
        \item Learnt how to specify grammar in yacc.
        \item Learnt to use yacc efficiently to to perform actions for each grammatical structure.
    \end{itemize}
\end{flushleft}
\hrule
\end{document}